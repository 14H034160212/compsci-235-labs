\documentclass{beamer}

\usepackage[utf8]{inputenc}
\usepackage{listings}
\usepackage{ulem}
\usepackage{hyperref}
\usepackage{amsmath}

\hypersetup{
    colorlinks=true,
    linkcolor=white,
    filecolor=magenta,      
    urlcolor=cyan,
}
 
\usetheme{Dresden}

\title[COMPSCI 235 Lab 1 (2020)]
{Installation/IDE/Debugging}
  
\author{Nisarag Bhatt}
 

\date[July 2020] 
{COMPSCI235 Lab 1}

\begin{document}
\frame{\titlepage}
\section{Introduction}
\begin{frame}
  \frametitle{Hello}
  My name is Nisarag and I am in 4th Year Software Engineering 
  \begin{itemize}
    \item Ask questions on Piazza instead of emailing me so your classmates can see the question and answers 
    \item These slides will be on Canvas, and any source code demonstrated along with TeX source code for these slides can be found on \href{https://github.com/FocalChord}{my GitHub}
  \end{itemize}
\end{frame}
\begin{frame}
  \begin{itemize}
  	\item Make a journal and keep note of what you did for each lab in this journal. Think of it as documenting your journey across the exercises across the lab
  	\item The journal may be digital or physical
  	\item At the end of the week, you will submit a pdf format of your journal
    \item We will be checking if your journal entry of that weeks lab is done properly 
  \end{itemize}
\end{frame}
\begin{frame}
  \begin{block}{Question}
  	Who doesn't have their own device here?
  \end{block}  
\end{frame}
\begin{frame}
  \frametitle{Learning Outcomes}
  \begin{itemize}
  	\item Install Python
    \item Understand Integrated Development Environments (IDEs)
    \item Configure PyCharm IDE for software development in Python programming language
    \item Create a Python based project in PyCharm IDE
    \item Write and debug a set of basic python programs in PyCharm IDE
    \item Virtual Environments / \texttt{requirements.txt}
  \end{itemize}
\end{frame}
\section{Python}
\begin{frame}
  \frametitle{Installation}
  \begin{block}{Question}
  	Who here has installed \texttt{Python 3}?    
  \end{block}
  \pause
  If you haven't, please download it from \href{https://www.python.org/downloads/}{here} 
\end{frame}
\section{IDE}
\begin{frame}
  \frametitle{PyCharm}
  \begin{itemize}
  	\item If you prefer using another IDE, text editor then feel free to use that in the course.
  	\item Jetbrains is a company which makes many IDE's for various languages. An IDE they have for Python is PyCharm.
    \item Download the Community Edition of PyCharm from \href{https://www.jetbrains.com/pycharm/}{here}
    \item If you want, you can also get the professional version by signing up to Jetbrains with your university email!
    \item Follow installation steps
   \end{itemize}
\end{frame}
\begin{frame}
  \frametitle{Demo}
  \begin{itemize}
  	\item Time for a demo with PyCharm and Debugging
  		\begin{itemize}
  			\item We will go over setting up a Hello World project. Trying doing the next 2 on your own.
  			\item Calculate and print factorial of a number
  			\item Print first 20 Fibonacci numbers
  		\end{itemize}
  	\item Code examples will be posted to Github
  \end{itemize}
   \end{frame}

\end{document}
